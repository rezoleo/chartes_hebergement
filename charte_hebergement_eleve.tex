\documentclass[12pt, a4paper]{article}

\usepackage[french]{babel}
\usepackage[T1]{fontenc}
\usepackage[utf8]{inputenc}
\usepackage{array, multirow, tabularx}
\usepackage{graphicx}
\usepackage{multicol}
\usepackage{xcolor}
\usepackage{colortbl}
\usepackage{url}

\usepackage{geometry}
\geometry{
  left=2cm,
  right=2cm,
  top=2cm,
  bottom=2cm
}

\usepackage{fontspec}
\setmainfont{OpenSans}[Path=./fonts/, Extension=.ttf,
  UprightFont=*-Regular,
  BoldFont=*-Bold,
  ItalicFont=*-Italic]

\begin{document}

\begin{center}
	\begin{minipage}{0.2\textwidth}
		\includegraphics[height=3cm]{rezoleo_logo.png}
	\end{minipage}
	\begin{minipage}{0.7\textwidth}
		\vspace{0.3cm}
		\Huge \textbf{Charte d'hébergement}\\
		\Huge \textbf{Individuel élève}
	\end{minipage}
\end{center}

\vspace*{0.5cm}

\hrule
\vspace{.5cm}
\noindent Nom : \makebox[8cm]{\dotfill}
\vspace{.5cm}
Prénom : \dotfill\\
Adresse e-mail : \dotfill\\
\hrule
\vspace{1cm}

\noindent Cette charte définit les règles de bonne utilisation de la solution d'hébergement proposée par le Rézoléo, et est établie entre l'élève indiqué ci-dessus, et l'association Rézoléo (déclarée le 19/05/2017 à la préfecture du Nord sous le RNA W595029406) et représentée par son président.

\section{Respect de la loi}

L'élève s'engage à respecter les dispositions légales en vigueur s'imposant à tout usager d'un système informatique, notamment mais sans se limiter à ce qui concerne le RGPD et la loi du 6 janvier 1978 relative à l'informatique, aux fichiers et aux libertés.

\section{Limites à l'utilisation}

Le service d'hébergement du Rézoléo ne peut pas être utilisé pour se livrer à des activités de spam, de publicité commerciale, ou d'activités illégales telles que les attaques \textit{DDOS} ou de \textit{Phishing}. De même, le service d'hébergement du Rézoléo ne peut pas être utilisé pour héberger des fichiers violant la propriété intellectuelle, le droit d'auteur ou le droit à l'image. La mise en place d'un logiciel \textit{serveur}, \textit{relais} ou similaire (p. ex : \textit{bouncer}, \textit{socks}, ...) est formellement interdite sans l'autorisation de Rézoléo.

\section{Conditions d'accès}
Le droit d'accès à l'espace de stockage est strictement réservé à l'élève. Ainsi, le mot de passe ne doit être communiqué en aucun cas. Ce compte est incessible et assorti de moyens d'identification. L'accès au compte peut être retiré si les conditions d'accès ne sont plus respectées ou si le comportement de l'utilisateur est contraire à la présente charte. Le responsable du compte doit prévenir Rézoléo de tout accès frauduleux ou tentative d'accès aux ressources qu'il utilise. Il est responsable de la protection de ses fichiers et de l'accès à ses données.

\section{Quota d'espace disque}
L'élève dispose d'un espace disque pour ses fichiers. Cette taille d'espace disque est limitée, et peut être changée par Rézoléo. Cette limite est par défaut fixée à \texttt{300Mo}.

\section{Confidentialité des données}
Les fichiers hébergés sont considérés comme privés. La lecture, la copie ou la modification d'un fichier ne peuvent être réalisées qu'après accord de l'élève, ou demande des autorités compétentes.

\section{Sanctions}
L'association Rézoléo est tenue par la loi de signaler toute action illégale constatée. Le non-respect de la présente charte ou de la loi peut entraîner la clôture anticipée du compte, ainsi que toute action légale que Rézoléo jugera nécessaire.

\section{Clôture du compte}
La clôture normale du compte s'effectue soit par demande de l'élève, soit 1 an après la fin d'étude à Centrale Lille Institut de l'élève.

\vspace*{2cm}

\noindent L'élève reconnait en signant avoir pris conscience du contenu de cette charte, des sanctions encourues, et s'engage à la respecter.

\vspace*{0.5cm}

\begin{flushright}
	\begin{minipage}{0.5\linewidth}
		Fait à Villeneuve-d'Ascq le : \dotfill\\
		Signature :
	\end{minipage}
\end{flushright}

\end{document}
